
\documentclass[12pt]{article}
\usepackage[paper=letterpaper,margin=2cm]{geometry}

\usepackage{amsmath}
\usepackage{amsthm} %needed for the proofs 
\usepackage{amssymb}

\usepackage{titling}
\usepackage{thmtools}
\usepackage{mathptmx} %font
\usepackage{verbatim} % for comments

%\usepackage{algpseudocode} % for algorithm pseudo code
%\usepackage{algorithm}
%\usepackage[linesnumbered,lined,boxed,commentsnumbered]{algorithm2e}
\usepackage[linesnumbered,ruled,vlined]{algorithm2e}
%macros for recursive functions
\newcommand{\forcond}{$i=0$ \KwTo $n$}
\SetKwFunction{FRecurs}{MaxSubArray}%
\SetKwProg{Fn}{Function }{\string:}{}
\SetKwData{Left}{left}\SetKwData{This}{this}\SetKwData{Up}{up}
\SetKwFunction{Union}{Union}\SetKwFunction{FindCompress}{FindCompress}
\SetKwInOut{Input}{input}\SetKwInOut{Output}{output}
%For plots
\usepackage{pgfplots}
\pgfplotsset{compat = newest}

\newtheorem{theorem}{Theorem}
\declaretheoremstyle{lemma}
\declaretheorem[style=lemma, name=Lemma]{lemma}

\theoremstyle{definition}
\newtheorem{definition}{Definition}

\declaretheoremstyle{example}
\declaretheorem[style=example, name=Example]{example}

\theoremstyle{remark}
\newtheorem*{remark}{Remark}

\declaretheoremstyle{proposition}
\declaretheorem[style=proposition, name=Proposition]{proposition}

\declaretheorem[name=Note]{note}
\declaretheoremstyle{note}

\newcommand\sol{%
  \\ 
  \\
  \textit{Solution:}\\%
}

\setlength{\droptitle}{-6em}

\title{\textsc{Assignment 1 -- Comp 252}}  
\author{Alexandre St-Aubin}
\date{\today}

\begin{document}
\maketitle 
\begin{enumerate}
  \item[\it Exercise 1] Consider a Fibonacci sequence starting with $x_0 = 0,\; x_1 = 1$. For positive integers $k$ and $n$, we would like to compute $x_n \mod k$ and are using the ram model of computation in which standard arithmetic operations, including “mod” take constant time. Describe how you would proceed in two cases: (1) $k = 627$, (2) $k = n$. In both cases, give your complexity in $O(\cdot)$.
  \sol  
  Case $(1)$: It is known that for any integer $k,$ the sequence of \textit{Fibonacci} numbers modulo $k$ has a period. Let $\pi (k)$, the \textit{Pisano Period, }denote the length of this period. A list of the \textit{Fibonacci} sequence modulo 627 of length $\pi (627)$ can be computed beforehand, since $k=627$ is constant. 

\IncMargin{1em}
\begin{algorithm}
\SetKwData{Left}{left}\SetKwData{This}{this}\SetKwData{Up}{up}
\SetKwFunction{Union}{Union}\SetKwFunction{FindCompress}{FindCompress}
\SetKwInOut{Input}{input}\SetKwInOut{Output}{output}
  \Input{The Pisano Period $pisanoPeriod$ for $k= 627$, an array $fibSeqMod$ of the first $pisanoPeriod$ integers of the Fibonacci sequence modulo 627, and the index $n$ of the Fibonacci sequence.}
\Output{$x_n \mod 627$}
\BlankLine
  {$x \leftarrow n \mod pisanoPeriod$}
\caption{Find the Fibonacci sequence number $x_n$ modulo 627. }\label{algo_disjdecomp}
  \KwRet{$fibSeqMod [x]$}
\end{algorithm}\DecMargin{1em}
  Note that in this first case, the Pisano period and in particular, the first $\pi (627)$ elements modulo $627$ of the Fibonacci sequence are constant, no matter the input $n$. It follows that the time complexity is 
  $$O(1), $$
  since the only thing our algorithm needs to compute is a modulo, and access an index of an array, both of which can be done in constant time in the \textsc{ram} model.

  
  Case (2): 

\newpage
\item[\it Exercise 2] In the bit model of computation, give an efficient algorithm for determining whether a given integer $n$ is a perfect square, and determine its worst-case complexity in big oh notation as a function of $n$.
\sol  
\IncMargin{1em}
\begin{algorithm}
\SetKwData{Left}{left}\SetKwData{This}{this}\SetKwData{Up}{up}
\SetKwFunction{Union}{Union}\SetKwFunction{FindCompress}{FindCompress}

\SetKwInOut{Input}{input}\SetKwInOut{Output}{output}
  \Input{An integer $n$.}
\Output{True or False.}
\BlankLine
$left \leftarrow 1$ \\ 
$right \leftarrow n$ \\ 
  \While{$left\leq right$}{
  $mid \leftarrow left + \lfloor(right-left)/2\rfloor $ \\ 
  $mid\_squared = mid* mid$\\
  \eIf{$n = mid\_squared$}{
  \Return True
  }{
  \uIf{$n<mid\_squared $}{
  $right \leftarrow mid-1$
  }\uElseIf{$n>mid\_squared $}{
  $left \leftarrow mid + 1$  
  }
  }
  }\Return False

\caption{Check whether an integer $n$ is a perfect square. }\label{algo_disjdecomp}
\end{algorithm}\DecMargin{1em}


We now determine the above algorithm's worst case time complexity in the bit model, denoting it $T_n$. In the worst case, line \textbf{3} will run $\log_2 n$ times, let $i$ be the iteration number. We safely estimate that in the worst case, at iteration $i,$ we have, $$|mid|= \lfloor \log_2 n \rfloor + 1$$ 
where $|\cdot |$ denotes the bit length of an integer. Now, we don't need to worry about the additions in each iteration, as they take linear time and are dwarfed by the multiplications/divisions, which take quadratic time. Hence, we conclude that the worst case time complexity in the bit model for this algorithm is 
$$T_n = O(\underbrace{\log_2 n}_{\text{\# of iterations}} \times \underbrace{(\log_2 n)^2}_{\text{multiplication}}) = O((\log n )^3) $$
\newpage 
\item[\it Exercise 3] Maximal Subarray problem. 
\sol  
\IncMargin{1em}
\begin{algorithm}

\SetKwArray{pre}{prefixTupleArray}
\SetKwArray{suf}{suffixTupleArray}
\SetKwArray{arr}{arr}
  \SetKwArray{maxTuple}{maxTuple}
\SetKwFunction{maxsubarray}{maxSubArray}
\caption{Maximum Subarray recursive, divide-and-conquer algorithm. }\label{algo_disjdecomp}
\Input{An array of integers \arr.}
\Output{A tuple representing the first and last index of the maximal subarray.}
  \pre, \suf $\leftarrow \arr;$  \\  
\tcp{\pre and \suf are arrays of tuples, \pre[i][0] contains \arr[i]}
\For{$i \leftarrow 1$ \KwTo $\text{length(\arr)}-1$}{
            \pre[i][0] += $\max(0, \text{\pre}[i-1])$;\\ 
            \eIf{\text{\pre}$[i-1]\geq 0$ }{
              \pre[i][1]$\leftarrow \pre[i-1][1]$;
            }{
            \pre[i][1] $\leftarrow i$;
            }
        }

\For{$i \leftarrow \text{length(\arr)}-2$ \KwTo $-1$}{
            \suf[i][0] += $\max(0, \text{\suf}[i+1])$\;
            \eIf{\text{\suf}$[i+1]\geq 0$ }{
              \suf[i][1] $\leftarrow \suf[i+1][1]$;
            }{
            \suf[i][1] $\leftarrow i$;
            }
        }
        \tcp{\pre[i][1] will hold the beginning index of the max subarray ending at i, and \suf[i][1], starting at i. }
    \BlankLine
  \Fn{\maxsubarray(array, left, right)}{
  \If{left = right}{
  \Return (array[left], (left, left)); \tcp{return a tuple}
  }
  $mid\leftarrow \lfloor (left + right)/2 \rfloor$;\\
  $ l \leftarrow \maxsubarray(array, left, mid)$;\\
  $r \leftarrow \maxsubarray(array, mid+1, right);$\\ 
  $ a \leftarrow (\pre[mid][0] + \suf[mid+1][0], (\pre[mid][1], \suf[mid+1][1]) ) $; \\   
  $\maxTuple \leftarrow l  $; \\
  \If{$\maxTuple[0]< r[0]$}{
   $\maxTuple \leftarrow r;$
  }
  \If{$\maxTuple[0]< a[0]$}{
   $\maxTuple \leftarrow a;$
  }

  \Return \maxTuple;
  }
\BlankLine 
  \Return \maxsubarray (\arr, 0, length(\arr))[1];\\

\end{algorithm}
\DecMargin{1em}



In the following algorithm, \textsc{prefixTupleArray}[i] is a tuple, with the first element being the maximal sum of a subarray \textbf{ending} at $i$, and the second being the index at which this subarray begins. On the other hand, \textsc{suffixTupleArray} will contain the maximal sum of a subarray \textbf{starting} at $i$ as its first element, and the index at which it ends as its second element. The complexity for the computation of prefix is always $O(n)$, as we must loop through each element of the array, similarly for suffix. 

The function \textsc{maxSubArray} is a divide-and-conquer algorithm that divides the array into smaller arrays, until only its length is of length 1. Then, it will compare each array that was divided in 2, and return a tuple representing the maximal sum within those two arrays at the first index, and another tuple with the beginning and ending indices of the subarray with maximal sum at the second index. There are three possibilities for the maximal subarray: 
\begin{enumerate}
  \item[\it (i)] The subarray is entirely contained in the \textbf{left} part of the parent array that was divided in 2;
  \item[\it (ii)] The subarray is entirely contained in the \textbf{right} part of the parent array that was divided in 2;
  \item[\it (iii)] The subarray is \textbf{split} between the two arrays on the left and the right. This is where the  \textsc{prefixTupleArray} and \textsc{suffixTupleArray} will be used. 
\end{enumerate}
The recurrence of the function described above is given by 
$$T_n = 2T_{n/2} + O(1). $$
We shall use the Master Theorem to compute its complexity. We notice that 
$$n^{\log_b (a)} / f(n) = n^{\log_2(2)}/1 = n, $$
and if we let $\varepsilon = 1/2$, then for any $n > 1$, we have $n > n^\varepsilon = \sqrt{n} {}$. Thus the first case of the theorem applies, and we conclude that
$$T_n = O(n^{\log_b (a)}) = O(n) .$$

\end{enumerate}
\end{document}
